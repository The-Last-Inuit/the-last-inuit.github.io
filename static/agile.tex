% Two-column Agile Lead Programmer Cheat Sheet
% Texifier-compatible (no special fonts required)

\documentclass[10pt,a4paper]{article}

\usepackage[margin=0.7in]{geometry}
\usepackage[T1]{fontenc}
\usepackage[utf8]{inputenc}
\usepackage{lmodern}
\usepackage{microtype}
\usepackage{parskip}
\usepackage{enumitem}
\usepackage{multicol}
\usepackage{titlesec}
\usepackage{hyperref}

\setlist[itemize]{leftmargin=*, itemsep=2pt, topsep=2pt}
\setlist[enumerate]{leftmargin=*, itemsep=2pt, topsep=2pt}

\titleformat{\section}{\large\bfseries}{}{0pt}{}
\titleformat{\subsection}{\normalsize\bfseries}{}{0pt}{}

\hypersetup{
  colorlinks=true,
  linkcolor=black,
  urlcolor=black
}

\begin{document}

\begin{center}
{\LARGE \textbf{Agile Lead Programmer}}\\
\vspace{4pt}
{\small Ship small value safely, continuously, with learning loops built in.}
\end{center}

\begin{multicols}{2}

\section{What ``nobody does agile'' usually means}
\begin{itemize}
  \item Standups become status reporting.
  \item Sprints become mini-waterfalls.
  \item Estimates become promises.
  \item Retros happen without consequences.
\end{itemize}

\textbf{My actual goal:} make reality visible, reduce risk early, keep flow unblocked, protect quality.

\section{My take}
\textit{Help the team ship small, valuable increments safely, continuously, with learning loops built in.}

\section{Six core habits}

\subsection{1) Make work thin and finishable}
\begin{itemize}
  \item Prefer items that complete in \textbf{1--3 days}.
  \item Slice by \textbf{workflow/value}, not by layers.
  \item Example: \textbf{Good} --- User can upload avatar (validation + UI feedback).
  \item Example: \textbf{Bad} --- Build avatar upload API (layer-only).
\end{itemize}

\textbf{Definition of Done (minimum):}
\begin{itemize}
  \item Code + tests
  \item Basic observability (logs/metrics)
  \item Notes/docs if behavior changes
  \item Feature flag if risky
\end{itemize}

\subsection{2) Protect flow with ruthless WIP limits}
\begin{itemize}
  \item Hard cap: \textbf{1 active item per dev}. (2 only if one is blocked/support.)
  \item Measure \textbf{cycle time} (start $\to$ done), not velocity.
  \item \textit{Almost done} counts as \textbf{not done}.
\end{itemize}

\subsection{3) Engineer quality}
Agile without quality becomes faster failure.
\begin{itemize}
  \item CI green is sacred.
  \item PRs small (aim \textbf{$<$300 lines changed}).
  \item Feature flags for risky work.
  \item Automated tests for critical paths.
  \item Anyone can block merge/deploy for safety.
\end{itemize}

\subsection{4) Run ceremonies as control systems}
\textbf{Standup (10 min):}
\begin{itemize}
  \item What moved to Done?
  \item What is blocked?
  \item What will move to Done today?
  \item If problem-solving starts: park it; 2 people continue after.
\end{itemize}

\textbf{Planning:}
\begin{itemize}
  \item Decide what you'll \textbf{finish}, not what you'll start.
  \item Readiness check: acceptance criteria, deps known, demoable outcome.
\end{itemize}

\textbf{Retro:}
\begin{itemize}
  \item Only matters if behavior changes.
  \item Pick \textbf{one} experiment for next iteration; track it visibly.
\end{itemize}

\textbf{Review/Demo:}
\begin{itemize}
  \item Demo increments to stakeholders.
  \item If nobody shows: you have a feedback problem.
\end{itemize}

\subsection{5) Make risk visible early (spikes \& seams)}
\begin{itemize}
  \item Time-boxed \textbf{spike} (0.5--2 days) for uncertainty.
  \item Create a \textbf{seam} (interface boundary) so risky parts can be swapped.
  \item Deliver a \textbf{walking skeleton}: thinnest end-to-end slice that proves integration.
\end{itemize}

\subsection{6) Build a truthful board}
If your board lies, your decisions will lie.
\begin{itemize}
  \item Columns: Backlog (ready) / In Progress / Review / Blocked / Done
  \item No item enters In Progress without clear acceptance criteria.
  \item Blocked means explicit: why, who, next action.
\end{itemize}

\section{Four metrics that actually help}
Skip maturity. Track:
\begin{enumerate}
  \item \textbf{Cycle time} --- how long items take once started
  \item \textbf{Throughput} --- items finished per week
  \item \textbf{WIP} --- items currently in progress
  \item \textbf{Change failure rate} --- deploys that break/rollback
\end{enumerate}

\textbf{Rules of thumb:}
\begin{itemize}
  \item High cycle time $\to$ slice smaller + reduce WIP.
  \item High change failure rate $\to$ invest in tests, flags, review quality, staging parity.
\end{itemize}

\section{Lead without becoming a Scrum Cop}
\begin{itemize}
  \item \textbf{Coach:} make work smaller/safer/clearer.
  \item \textbf{Shield:} protect focus from random interrupts.
  \item \textbf{Amplifier:} show stakeholders reality early/often.
  \item \textbf{Systems thinker:} fix bottlenecks, not people.
\end{itemize}

Useful phrases:
\begin{itemize}
  \item What would \textit{done} look like by tomorrow?
  \item What's the smallest demoable slice?
  \item What's the risk, and how do we surface it this week?
  \item What do we stop doing to make room for this?
\end{itemize}

\section{Two-week practical plan}

\subsection{Week 1: Stabilize flow}
\begin{itemize}
  \item Enforce \textbf{WIP = 1}.
  \item Start tracking \textbf{cycle time}.
  \item Slice big tasks into 1--3 day increments.
  \item Set PR-size and CI discipline expectations.
\end{itemize}

\subsection{Week 2: Improve learning and safety}
\begin{itemize}
  \item Feature-flag habit for risky changes.
  \item Retro selects 1 experiment; you track it.
  \item Stakeholder demo to gather feedback.
\end{itemize}

\vfill
\columnbreak

\section{Quick diagnosis template}
\begin{itemize}
  \item Team size and stack
  \item Release cadence (daily/weekly/monthly)
  \item Biggest pain: interrupts / late surprises / slow delivery / regressions / unclear priorities
\end{itemize}

\section{Working Agreements}
\begin{itemize}
  \item We optimize for \textbf{finished work}, not started work.
  \item WIP is limited (default: 1 item per dev).
  \item CI must be green before merge.
  \item PRs stay small; if large, explain why.
  \item Risky work uses feature flags.
  \item ``Blocked'' always has an owner and next action.
  \item Retros produce one measurable experiment.
\end{itemize}

\section{Board policy}
\begin{itemize}
  \item Backlog items are "ready" only with clear acceptance criteria and known dependencies. 
  \item Work enters In Progress only when someone is available to carry it through to Done. 
  \item Blocked items explicitly state why, who can unblock, and the next action. 
  \item Done means code is merged, tested, and deployed (or marked explicitly as "done-not-deployed").
\end{itemize}

\end{multicols}

\end{document}
